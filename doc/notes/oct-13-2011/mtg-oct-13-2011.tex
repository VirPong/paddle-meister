% !TEX TS-program = pdflatex
% !TEX encoding = UTF-8 Unicode

% This is a simple template for a LaTeX document using the "article" class.
% See "book", "report", "letter" for other types of document.

\documentclass[11pt]{article} % use larger type; default would be 10pt

\usepackage[utf8]{inputenc} % set input encoding (not needed with XeLaTeX)

%%% Examples of Article customizations
% These packages are optional, depending whether you want the features they provide.
% See the LaTeX Companion or other references for full information.

%%% PAGE DIMENSIONS
\usepackage{geometry} % to change the page dimensions
\geometry{letterpaper} % or letterpaper (US) or a5paper or....
% \geometry{margins=2in} % for example, change the margins to 2 inches all round
% \geometry{landscape} % set up the page for landscape
%   read geometry.pdf for detailed page layout information

%\usepackage{graphicx} % support the \includegraphics command and options

% \usepackage[parfill]{parskip} % Activate to begin paragraphs with an empty line rather than an indent

%%%% PACKAGES
%\usepackage{booktabs} % for much better looking tables
%\usepackage{array} % for better arrays (eg matrices) in maths
%\usepackage{paralist} % very flexible & customisable lists (eg. enumerate/itemize, etc.)
%\usepackage{verbatim} % adds environment for commenting out blocks of text & for better verbatim
%\usepackage{subfig} % make it possible to include more than one captioned figure/table in a single float
%% These packages are all incorporated in the memoir class to one degree or another...
%
%%%% HEADERS & FOOTERS
%\usepackage{fancyhdr} % This should be set AFTER setting up the page geometry
%\pagestyle{fancy} % options: empty , plain , fancy
%\renewcommand{\headrulewidth}{0pt} % customise the layout...
%\lhead{}\chead{}\rhead{}
%\lfoot{}\cfoot{\thepage}\rfoot{}
%
%%%% SECTION TITLE APPEARANCE
%\usepackage{sectsty}
%\allsectionsfont{\sffamily\mdseries\upshape} % (See the fntguide.pdf for font help)
%% (This matches ConTeXt defaults)
%
%%%% ToC (table of contents) APPEARANCE
%\usepackage[nottoc,notlof,notlot]{tocbibind} % Put the bibliography in the ToC
%\usepackage[titles,subfigure]{tocloft} % Alter the style of the Table of Contents
%\renewcommand{\cftsecfont}{\rmfamily\mdseries\upshape}
%\renewcommand{\cftsecpagefont}{\rmfamily\mdseries\upshape} % No bold!
%
%%%% END Article customizations

%%% The "real" document content comes below...

\title{Meeting Notes}
\author{Server Group}
\date{13 Oct 2011} % Activate to display a given date or no date (if empty),
         % otherwise the current date is printed

\begin{document}
\maketitle

\section{Members Attended}

\begin{itemize}
        \item Shelby Lee (M)
        \item Ryan Wheeler
        \item Patrick Green
        \item Daniel Guilak
\end{itemize}

\section{Review of Goals}

\begin{itemize}
         \item Work on phone to server, server to database (Mongo), database to sever and web communications. - all
       \item Understanding websockets - all
        \item Querying the database - Shelby
        \item Finding and understanding javascript ping pong game - Dan, Patrick, Ryan


\end{itemize}

\section{What has been accomplished}

\begin{itemize}
        \item Connection: currently phone-to-server one-way conversations. Multiple phones send data to server and server sends that data back plus some extra data. Database was having some trouble connoting (using javascript very ugly), downloaded node-mongodb-native to help facilitate connection code.
       \item Found and are currently in the process of tearing apart a ping pong game to help build game logic.
        \item Understand how to query from mongodb, in the process of creating javascript code
\end{itemize}

\section{New Goals}

\begin{itemize}
        \item Connection: have the phone-server hold a two-way conversation - Dan
       \item Build ping pong game logic (possibly using Garret's code) - Patrick, Ryan
       \item Have the database connect and communicate to the server to take queries - Shelby
\end{itemize}

\end{document}