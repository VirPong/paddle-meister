% !TEX TS-program = pdflatex
% !TEX encoding = UTF-8 Unicode

% This is a simple template for a LaTeX document using the "article" class.
% See "book", "report", "letter" for other types of document.

\documentclass[11pt]{article} % use larger type; default would be 10pt

\usepackage[utf8]{inputenc} % set input encoding (not needed with XeLaTeX)

%%% Examples of Article customizations
% These packages are optional, depending whether you want the features they provide.
% See the LaTeX Companion or other references for full information.

%%% PAGE DIMENSIONS
\usepackage{geometry} % to change the page dimensions
\geometry{letterpaper} % or letterpaper (US) or a5paper or....
% \geometry{margins=2in} % for example, change the margins to 2 inches all round
% \geometry{landscape} % set up the page for landscape
%   read geometry.pdf for detailed page layout information

%\usepackage{graphicx} % support the \includegraphics command and options

% \usepackage[parfill]{parskip} % Activate to begin paragraphs with an empty line rather than an indent

%%%% PACKAGES
%\usepackage{booktabs} % for much better looking tables
%\usepackage{array} % for better arrays (eg matrices) in maths
%\usepackage{paralist} % very flexible & customisable lists (eg. enumerate/itemize, etc.)
%\usepackage{verbatim} % adds environment for commenting out blocks of text & for better verbatim
%\usepackage{subfig} % make it possible to include more than one captioned figure/table in a single float
%% These packages are all incorporated in the memoir class to one degree or another...
%
%%%% HEADERS & FOOTERS
%\usepackage{fancyhdr} % This should be set AFTER setting up the page geometry
%\pagestyle{fancy} % options: empty , plain , fancy
%\renewcommand{\headrulewidth}{0pt} % customise the layout...
%\lhead{}\chead{}\rhead{}
%\lfoot{}\cfoot{\thepage}\rfoot{}
%
%%%% SECTION TITLE APPEARANCE
%\usepackage{sectsty}
%\allsectionsfont{\sffamily\mdseries\upshape} % (See the fntguide.pdf for font help)
%% (This matches ConTeXt defaults)
%
%%%% ToC (table of contents) APPEARANCE
%\usepackage[nottoc,notlof,notlot]{tocbibind} % Put the bibliography in the ToC
%\usepackage[titles,subfigure]{tocloft} % Alter the style of the Table of Contents
%\renewcommand{\cftsecfont}{\rmfamily\mdseries\upshape}
%\renewcommand{\cftsecpagefont}{\rmfamily\mdseries\upshape} % No bold!
%
%%%% END Article customizations

%%% The "real" document content comes below...

\title{Meeting Notes}
\author{Server Group}
\date{27 Oct 2011} % Activate to display a given date or no date (if empty),
         % otherwise the current date is printed

\begin{document}
\maketitle

\section{Members Attended}

\begin{itemize}
        \item Shelby Lee (M)
        \item Ryan Wheeler
        \item Patrick Green
        \item Daniel Guilak
\end{itemize}

\section{Review of Goals}

\begin{itemize}
         \item Connection: have the phone-server hold a two-way conversation - Dan
       \item Building pong game logic - Patrick, Ryan
        \item Having the database connect and communicate to the server to take queries


\end{itemize}

\section{What has been accomplished}

\begin{itemize}
        \item Connection: A pair of phones can now communicate through the server. At the moment, the current interface is very clunky. 
	\item Database: there has been no further progress in code for the javascript as Shelby was reassigned from this secondary feature to help Daniel oversee connection for an upcoming demo and finish the intermediate report.
	\item Game logic: code is near-finished; two paddles react on movement to key presses. Temporary freezing to ensure all members participated in finishing the intermediate report.
\end{itemize}

\section{New Goals}

\begin{itemize}
        \item Connection: polish communication code to make runtime smoother - Dan, Shelby
       \item Game Logic: finish building a functional and rudimentary pong game logic model by code freeze- Patrick, Ryan
       \item Database: after phone-server-phone communication is polished, return to building database connection code with server to take queries
\end{itemize}

\end{document}