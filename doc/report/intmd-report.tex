\documentclass[letterpaper,12pt]{article}
\usepackage{geometry}
\geometry{paper=letterpaper,
                       body={6.50in, 9.00in},
                       lmargin=1.00in,
%                       rmargin=0.75in,
                       vmarginratio={1:1}
}
\begin{document}
\title{Intermediate Report -- Server Group}
\author{Ryan Wheeler, Shelby Lee, Patrick Green, Daniel Guilak}
\date{\today}
\maketitle

\tableofcontents

\section{Team Profile}

\section{Functional Requirements}
\subsection{Use Cases}
%Will probably have lots of \newpage commands here.
\subsection{System Sequence Diagram}
\section{Nonfunctional Requirements}
\section{Domain Analysis}
\section{Implementation}
\subsection{Style Guide}
\subsubsection{Whitespace}
The basic indentation is two spaces. Tabs are not to be used at all.
Try to keep lines to 80 characters or less. When wrapping lines, try to indent to line up with a related item on the previous line. Examples: 
\begin{verbatim}
var result = prompt(aMessage,
                    aInitialValue,
                    aCaption);
\end{verbatim}
Lines should not contain trailing spaces, even after binary operators, commas or semicolons.
Separate binary operators with spaces.
Spaces go after commas and semicolons, but not before.
Spaces go after keywords. For example:
\begin{verbatim}
if (x > 0)
\end{verbatim}
One (or two) blank lines between block definitions. Also consider breaking up large code blocks with blank lines.

\subsubsection{Commenting}
Use multi-line comments to explain the purpose behind blocks of code.
Use in-line comments often to clarify actions.

\subsubsection{Symbols}
Function braces can either be on their own line or the open brace on the function declaration, i.e.

\begin{verbatim}
function toOpenWindow(aWindow)
  {
    aWindow.document.commandDispatcher.focusedWindow.focus();  
  }

function toOpenWindow(aWindow) {
    aWindow.document.commandDispatcher.focusedWindow.focus();  
}
\end{verbatim}
Spaces are not necessary inside brackets e.g. parameter lists, array subscripts.
Prefer double quotes, except in in-line event handlers or when quoting with double quotes ie (‘he said, “trouble” ’).
Have braces indented relative to their parent statement.
\subsubsection{Code Style}
Always put the else of an if-else statement on its own line.
Both i++ and ++i should not be used.
\subsubsection{Function and Varible Naming}
Function names, local variables and object members have no prefix.
Try to declare local variables as near to their use as possible; try to initialize every variable.

\subsubsection{JavaScript Features}
Make sure that your code doesn't generate any strict JavaScript warnings, such as:
\begin{itemize}
      	  \item{Duplicate variable declaration}
      	  \item{Mixing return; with return value;}
      	  \item{Trailing comma in JavaScript object declarations}
\end{itemize}    	
If you are unsure if an array value exists, compare the index to the array's length. If you are unsure if an object member exists, use "name" in aObject, or if you are expecting a particular type you may use typeof aObject.name == "function" (or whichever type you are expecting).
Use { member: value, ... } to create a JavaScript object; a useful advantage over new Object() is the ability to create initial properties and use extended JavaScript syntax to define getters and setters.  If having defined a constructor you need to assign default properties it is preferred to assign an object literal to the prototype property. For example,

\begin{verbatim}
function SupportsString(data)
{
  this.data = data;
}
\end{verbatim}
Do not compare booleans to true or false. For example, write if (ioService.offline). Compare objects to null, numbers to 0 or strings to "" if there is chance for confusion.
\subsection{Install Documentation}
\subsubsection{Server}
Requirements:
\begin{itemize}
\item A relatively new machine (read: purchased/built in the last six years or so) running Linux or Mac OS X (These instructions were generated on a machine running Ubuntu 11.10 -- the configuration used for generation of instructions will henceforth be denoted in parentheses).
\item Basic Unix tools will already be installed, but other requirements that may not be present in a standard installation are:
Curl (using 7.21.6)
Git (using 1.7.5.4)
\item Connection to the Internet. 
\end{itemize}
Instructions:
Each line is a unix command, and so each of these commands must be typed into a shell of some sort, such as BASH or ZSH (using BASH). Alternatively, these commands can be easily generated into a BASH shell script quite easily of so desired.

To install node and the node package manager (npm):

\begin{verbatim}
echo 'export PATH=$HOME/local/bin:$PATH' >> ~/.bashrc
. ~/.bashrc
mkdir ~/local
mkdir ~/node-v0.4.9
cd ~/node-v0.4.9
curl http://nodejs.org/dist/node-v0.4.9.tar.gz | tar xz --strip-components=1
./configure --prefix=~/local
make install # ok, fine, this step probably takes more than 30 seconds...
curl http://npmjs.org/install.sh | sh
\end{verbatim}

To install the express web framework and socket.io:

\begin{verbatim}

npm install -g express socket.io #installs express server (expressjs.org) and socket.io
cd <your directory name>
express express-test && cd express-test
npm install -d #resolve dependencies
node app.js #runs test server with node
#point browser to localhost:3000 (default) -- should see welcome message
\end{verbatim}

To run the server:

First, clone the paddle-meister repository using git:

\begin{verbatim}
git clone git://github.com/VirPong/paddle-meister.git
\end{verbatim}

Then, navigate to the socket-test folder and execute

\begin{verbatim}
node server.js
\end{verbatim}

The server will run at whichever port specified in the PORT variable in server.js. To shutdown the server, press CTRL-C twice.

\subsubsection{MongoDB and Modules}
First go to mongodb.org and find the path to the most recent 64-bit Linux version. Then put this onto the server using shell commands:

\begin{verbatim}
wget [path copied from web]
\end{verbatim}
Then uncompress the file using tar with

\begin{verbatim}
tar zxvf [file name].tgz
\end{verbatim}

Then install MongoDB using the synaptic package manager (Debian-based Linux distributions only):
\begin{verbatim}
sudo apt-get install mongodb
\end{verbatim}

After looking over this open-source module, we determined that node-mongodb-native (https://github.com/christkv/node-mongodb-native) was perfectly suited for our project. This module facilitates a connection between mongodb and our server, which uses node. However, we ran into a compilation error when attempting to run example code, complaining of the native bson parser not being compiled.
In searching for a solution, we were pointed to another module for mongodb, called mongojs (https://github.com/gett/mongojs). Mongojs is actually a wrapper for mongodb-native that allows code to be written in a format almost identical to the mongodb shell API. This allows the code to be condense and clear.
We retrieved node-mongodb-native from github and mongojs was installed via npm with the shell command:
\begin{verbatim}
npm install mongojs
\end{verbatim}

\subsection{Algorithms, Data Structures, and Design Patterns}
\subsection{Data Storage}
\subsection{Testing and Verification}
\section{Planning and Reflection}
\subsection{Schedule}
\subsection{Challenges}
\subsection{Future milestones and Member Assignments}
\section{References}
\subsection{Git and Github}
	\begin{itemize}
		\item Git (http://git-scm.com/) is a distributed version control system that is widely-used in the open source community.
		\item Github (http://github.com/) is a web-based hosting service for projects using git. It provides collaboration tools and many other features such as pull requests and commit management that helps small development groups work efficiently.
		\item The current project repository is hosted on github and collaboration will happen through the web-based UI.
	\end{itemize}
\subsubsection{node.js}
	\begin{itemize}
		\item node.js (http://nodejs.org/) is an easy-to-learn server-side JavaScript server development environment that supports a myriad of different communication protocols such as HTTP, TCP, and WebSockets.
		\item Will be used as the framework for the server.
	\end{itemize}
\subsubsection{MongoDB}
	\begin{itemize}
		\item MongoDB (http://mongodb.org/) is a scalable open-source non-relational database with strong node.js support.
		\item Will be used as the main database for storing and retrieving game information.
	\end{itemize}
\subsubsection{Eclipse}
	\begin{itemize}
		\item Eclipse is an Integrated Development Environment written in Java that provides support for many different programming languages and version control systems through plugins -- A git plugin is available, called Egit (http://eclipse.org/egit/).
		\item Most team members will be using Eclipse IDE for development.
	\end{itemize}
\subsubsection{Google Docs}
	\begin{itemize}
		\item Google Docs (http://docs.google.com/) is an online document processor with tools similar to Microsoft Word, Powerpoint, and others. It allows for collaboration between different users concurrently on the same document.
		\item Will be used for collaboration on internal documents and documentation.
	\end{itemize}
\section{Schedule}
\begin{itemize}
	\item Oct 7 -- A database implemented on the server.
	\begin{itemize}
		\item  Be able to have the server do a simple query from the mongoDB database and display it.
	\end{itemize}
	\item Oct 7 -- Get the server up and running. 
	\begin{itemize}
		\item Be able to have a server that provides text or visual response to simple sample packets from other computers and relays sample information from the database.
	\end{itemize}
	\item Oct 11 -- Basic server communication to other devices. 
	\begin{itemize}
		\item Have a proof of concept for devices on the Android and iOS platforms to be able to communicate with the server. 
		\item Have the clients be able to display information queried from the mongoDB database by communicating with the server.
	\end{itemize}
	\item Oct 17 -- Simple game engine/logic working.
	\begin{itemize}
		\item Translate the rules of two-player (non-computer opponents) pong into JavaScript code that the server can use.
		\item Most likely will test on a browser such as Chrome or Chromium first.
	\end{itemize}
	\item Nov 4 -- Real-time game streaming on web browsers.
	\begin{itemize}
		\item Be able to watch a game that is happening between two other clients on a third client (most likely a browser at first
	\end{itemize}
\end{itemize}
\end{document}
